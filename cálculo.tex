\documentclass[addpoints]{article}
\usepackage[brazilian]{babel}
\usepackage[utf8]{inputenc}
\usepackage[export]{adjustbox}
\usepackage{hyperref}
\usepackage{float}
\usepackage{amssymb,amsthm,amstext,latexsym,amsmath,fullpage,graphicx}

\section{Introdução}
Este trabalho tem por finalidade esplanar de forma sucinta as propriedades analíticas e algébricas da função complexa $f(z)=z^{\lambda} , \forall \lambda \in \mathbb{C}$, então inicialmente vamos definir a função de variavel complexa, a definição a seguir pode ser encontrada no livro de Ávila (2003 p.34) 
\subsection{Função de variável complexa}
Seja $D$ um subconjunto de $\mathbb{C}$. Uma função complexa $f: D \longrightarrow \mathbb{C}$ é uma lei que a cada elemento de $z$ de $D$ associa um unico elemento complexo $w=f(z) \in \mathbb{C}$. 
\subsection{Definição de função unívoca e plurívoca}
Uma função é dita unívoca se dados $z_1, z_2\in \mathbb{C}$
tal que  $z_1 \neq z_2 \Rightarrow f(z_1) \neq f(z_2)$ ,similarmente funções plurívocas, multiformes ou multivalentes são funções que, para um valor dado da variável z, associam dois ou mais números  $w=f(z)$
